\section{原子核物理概论}
\subsection{三种射线的性质}
\begin{enumerate}
\item $\alpha$: He核束. 电离能力最强, 穿透能力最弱. 
\item $\beta$: 电子束. 
\item $\gamma$: 频率极高(高于X光)的电磁波. 穿透能力最强, 电离能力最弱. 
\end{enumerate}
$\alpha$, $\beta$, $\gamma$三种射线电离性依次减弱, 穿透性依次增强. 

\subsection{裂变和聚变}
\begin{itemize}
\item \textbf{裂变}: 用慢中子(为什么要用慢中子?)轰击轴235的反应式(下式只是其中一种形式, 还有其它的裂变方式)
\[
\textrm{n}+^{235}\textrm{U}\longrightarrow^{236}\textrm{U}^*\longrightarrow^{144}\textrm{Ba}+^{89}\textrm{Kr}+3\textrm{n}
\]
美国在日本广岛和长崎爆炸的两颗原子弹分别以$^{235}\textrm{U}$和$^{239}\textrm{Pu}$为燃料. 
$^{235}\textrm{U}$只占天然铀0.72\%, $^{238}\textrm{U}$占天然铀99.27\%, 
虽然$^{238}\textrm{U}$不能直接应用, 却可以用它来生产有用的核燃料, 下式是$^{238}\textrm{U}$反应制$^{239}\textrm{Pu}$
\[
\textrm{n}+^{238}\longrightarrow^{239}\textrm{U}+\gamma
\]
\[
^{239}\textrm{U}\longrightarrow^{239}\textrm{Np}+\textrm{e}^-+\overline{\nu}_e
\]
\[
^{239}\textrm{Np}\longrightarrow^{239}\textrm{Pu}+\textrm{e}^-+\overline{\nu}_e
\]
\item \textbf{聚变}: 氢弹的反应属于轻核聚变, 能放出巨大的能下, 反应式
\[
\textrm{d}+\textrm{T}\longrightarrow\alpha+\textrm{n}+17.58\textrm{MeV}
\]
氘(d)在自然界很长见, 但氚(T)在自然界中却不存在, 但可能通过下式反应制得
\[
\textrm{n}+^6\textrm{Li}\longrightarrow\alpha+\textrm{T}+4.9\textrm{MeV}
\]
\end{itemize}
